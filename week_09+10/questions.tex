\documentclass{exam}

\usepackage[margin=1in]{geometry}
\usepackage{amsmath}
\usepackage{amssymb}
\usepackage{amsthm}
\usepackage{mathtools}
\usepackage{bm}

\usepackage{color}
\usepackage{colortbl}
\definecolor{deepblue}{rgb}{0,0,0.5}
\definecolor{deepred}{rgb}{0.6,0,0}
\definecolor{deepgreen}{rgb}{0,0.5,0}
\definecolor{gray}{rgb}{0.7,0.7,0.7}

\usepackage{hyperref}
\hypersetup{
  colorlinks   = true, %Colours links instead of ugly boxes
  urlcolor     = black, %Colour for external hyperlinks
  linkcolor    = blue, %Colour of internal links
  citecolor    = blue  %Colour of citations
}

\usepackage{listings}
\lstset {
	basicstyle=\ttfamily,
    ,language=SQL
    ,showstringspaces=false
    ,keepspaces=true
}

%%%%%%%%%%%%%%%%%%%%%%%%%%%%%%%%%%%%%%%%%%%%%%%%%%%%%%%%%%%%%%%%%%%%%%%%%%%%%%%%

%\printanswers

% Create a True False question format
\newcommand*{\TrueFalse}[1]{%
\ifprintanswers
    \ifthenelse{\equal{#1}{T}}{%
        \textbf{TRUE}\hspace*{14pt}False
    }{
        True\hspace*{14pt}\textbf{FALSE}
    }
\else
    {True}\hspace*{20pt}False
\fi
} 
%% The following code is based on an answer by Gonzalo Medina
%% https://tex.stackexchange.com/a/13106/39194
\newlength\TFlengthA
\newlength\TFlengthB
\settowidth\TFlengthA{\hspace*{1.16in}}
\newcommand\TFQuestion[2]{%
    \setlength\TFlengthB{\linewidth}
    \addtolength\TFlengthB{-\TFlengthA}
    \parbox[t]{\TFlengthA}{\TrueFalse{#1}}\parbox[t]{\TFlengthB}{#2}
}


\theoremstyle{definition}
\newtheorem{problem}{Problem}
\newtheorem{defn}{Definition}
\newcommand{\R}{\mathbb R}
\DeclareMathOperator{\E}{\mathbb E}
\DeclareMathOperator{\nnz}{nnz}
\DeclareMathOperator{\determinant}{det}
\DeclareMathOperator{\Var}{Var}
\DeclareMathOperator{\rank}{rank}
\DeclareMathOperator*{\argmin}{arg\,min}
\DeclareMathOperator*{\argmax}{arg\,max}

\newcommand{\I}{\mathbf I}
\newcommand{\Q}{\mathbf Q}
\newcommand{\p}{\mathbf P}
\newcommand{\pb}{\bar {\p}}
\newcommand{\pbb}{\bar {\pb}}
\newcommand{\pr}{\bm \pi}

\newcommand{\trans}[1]{{#1}^{T}}
\newcommand{\loss}{\ell}
\newcommand{\w}{\mathbf w}
\newcommand{\x}{\mathbf x}
\newcommand{\y}{\mathbf y}
\newcommand{\lone}[1]{{\lVert {#1} \rVert}_1}
\newcommand{\ltwo}[1]{{\lVert {#1} \rVert}_2}
\newcommand{\lp}[1]{{\lVert {#1} \rVert}_p}
\newcommand{\linf}[1]{{\lVert {#1} \rVert}_\infty}
\newcommand{\lF}[1]{{\lVert {#1} \rVert}_F}

% Shalev-Swartz and Ben-David

\newcommand{\h}{\mathcal H}
\newcommand{\D}{\mathcal D}
\DeclareMathOperator*{\erm}{ERM}

%%%%%%%%%%%%%%%%%%%%%%%%%%%%%%%%%%%%%%%%%%%%%%%%%%%%%%%%%%%%%%%%%%%%%%%%%%%%%%%%

\begin{document}


\begin{center}
\Huge
Questions on Indexes and Table Storage
\end{center}


\section{True/False Questions}

For each question below, circle either True or False.
On your final exam,
each correct answer will result in +1 point,
each incorrect answer will result in -1 point,
and each blank answer in 0 points.
For this homework assignment, you can uncomment the following line in the tex file to view the answers:
\begin{verbatim}
\printanswers
\end{verbatim}
and so these questions do not need to be submitted.
You should still try to complete them, however, to check your understanding.
Approximately 4/5 of these questions are answered in class,
and the remaining 1/5 you'll have to refer to the postgres documentation / supplementary material for the answers.

\begin{questions}

%%%%%%%%%%%%%%%%%%%%%%%%%%%%%%%%%%%%%%%%%%%%%%%%%%%%%%%%%%%%%%%%%%%%%%%%%%%%%%%%
\fullwidth{\textbf{Table Storage}}

\question\TFQuestion{T}{A table that takes up 160KB on disk has 20 pages.}

\question\TFQuestion{F}{TID stands for \emph{Transaction ID}, and every transaction is assigned a unique TID.}

\question\TFQuestion{T}{OID stands for \emph{Object ID}, and every table is assigned a unique OID.}

\question\TFQuestion{F}{There is a hard limit of 100 tuples per page.}

\question\TFQuestion{F}{Very large tuples can span multiple pages.}

\question\TFQuestion{F}{If a tuple exists on disk, then there is guaranteed to be some transaction that can see the tuple.}

\question\TFQuestion{F}{In order to remove dead tuples from a table, you must manually run the VACUUM command on that table.}

\question\TFQuestion{F}{You should disable autovacuum to improve the performance of your database.}

\question\TFQuestion{F}{When you use the INSERT command to insert multiple rows into a table at once, these rows are guaranteed to be inserted into the same page.}

\question\TFQuestion{F}{A postgres database cluster can span dozens of computers.}

\question\TFQuestion{F}{Postgres is using too much disk space, and you need to free up some space.  You identify that there is a large 10TB table that contains about 90\% dead tuples.  Running the VACUUM command on this table will free up several terabytes of disk space.}

\question\TFQuestion{T}{Postgres uses a table's FSM to quickly determine which page it should insert a tuple into.}

\question\TFQuestion{T}{Postgres uses a table's VM to speed up VACUUMing.}

\question\TFQuestion{F}{If you are inserting large text strings into postgres, it makes sense to compress them first in order to save space.}

\question\TFQuestion{F}{The DELETE command deletes tuples directly from the page files.}

\question\TFQuestion{F}{NoSQL databases like MongoDB and CassandraDB use ACID compliant transactions.}

\question\TFQuestion{T}{Postgres transactions are ACID compliant, and therefore slower than non-ACID databases.}

\question\TFQuestion{T}{The \lstinline{synchronous_commit} system setting can be used to disable postgresql's ACID guarantees.  This speeds up transactions, but may result in data loss if the server crashes.}

\question\TFQuestion{F}{If the write ahead log (WAL) grows very large, it is safe to delete it in order to free up disk space.}

\question\TFQuestion{F}{If the transaction log (clog/xact) grows very large, it is safe to delete it in order to free up disk space.}

\question\TFQuestion{T}{Given the choice between (A) having tables on an SSD and indexes on an HDD, and (B) having tables on an HDD and indexes on an SDD, option (A) will generally be faster.}
\question\TFQuestion{T}{If the database cluster is being stored on an SSD, then the \lstinline{random_page_cost} system parameter should should be reduced from its default value of 4.}
\question\TFQuestion{T}{The default \lstinline{fillfactor} for tables is 100, and for btree indexes is 90.}
\question\TFQuestion{T}{Tables that have INSERTs but no UPDATEs should use a \lstinline{fillfactor} of 100, but for tables with many updates, it may make sense to decrease the \lstinline{fillfactor}.}
\question\TFQuestion{F}{The order of columns in a multicolumn index has no effect on which queries the index can speed up.}
\question\TFQuestion{T}{Multicolumn indexes will have a smaller fanout than a single column index created on any of the indexes columns.}
\question\TFQuestion{F}{The advantage of using a partial index is that it saves disk space, but the disadvantage is that more pages are accessed during an index only, index, or bitmap scan.}



%%%%%%%%%%%%%%%%%%%%%%%%%%%%%%%%%%%%%%%%%%%%%%%%%%%%%%%%%%%%%%%%%%%%%%%%%%%%%%%%
\fullwidth{\textbf{BTree}}

\question\TFQuestion{F}{Increasing the fanout of a BTree will increase the depth of the tree.}
\question\TFQuestion{T}{For all integers $B,n \ge2$, it is true that $B\log_B n \ge \log_2 n$.}
\question\TFQuestion{T}{Postgres uses B+ Trees in the BTree index.}
\question\TFQuestion{F}{When analyzing the performance of an algorithm on BTrees, the most import metric to consider is the number of comparision operations.}
\question\TFQuestion{T}{On a typical HDD, seeks are expensive operations.}
\question\TFQuestion{F}{Balanced binary search trees like the AVL Tree or Red-Black Tree tend to perform better than BTrees for large datasets that cannot fit in memory, and must be stored on disk.}
\question\TFQuestion{T}{You have created a BTree index on an INTEGER column.  The fanout of the tree will typically be in the hundreds.}
\question\TFQuestion{F}{All nodes in a BTree, as implemented in postgres, are guaranteed to have the same fanout.}
\question\TFQuestion{F}{Binary trees typically have a lower height than BTrees.}
\question\TFQuestion{T}{B+ Trees support faster range queries than BTrees.}
\question\TFQuestion{T}{For HDDs with a very slow seek time but fast sequential read time, it makes sense to have higher fanout when using a BTree.}
\question\TFQuestion{T}{NULL values are stored in postgresql indexes by default.}


%%%%%%%%%%%%%%%%%%%%%%%%%%%%%%%%%%%%%%%%%%%%%%%%%%%%%%%%%%%%%%%%%%%%%%%%%%%%%%%%
\fullwidth{\textbf{Scan Algorithms}}

\question\TFQuestion{F}{An index can be used to speed up every slow query.}
\question\TFQuestion{F}{Creating an indexes on a table will make INSERT statements faster on that table.}
\question\TFQuestion{F}{When it is possible to perform both an index scan and a bitmap index scan, the bitmap index scan is guaranteed to be faster.}
\question\TFQuestion{T}{When it is possible to perform both an index only scan and a bitmap index scan, the index only scan is guaranteed to be faster.}
\question\TFQuestion{T}{When it is possible to perform both an index only scan and an index scan, the index only scan is guaranteed to be faster.}
\question\TFQuestion{F}{When it is possible to perform both an index scan and sequential scan, the index scan is guaranteed to be faster.}
\question\TFQuestion{F}{When it is possible to perform both a bitmap index scan and sequential scan, the bitmap index scan is guaranteed to be faster.}
\question\TFQuestion{T}{Bitmap scans are the only scan method that can take advantage of multiple indexes.}
\question\TFQuestion{F}{When it is possible to perform both an index only scan and sequential scan, the index only scan is guaranteed to be faster.}
\question\TFQuestion{F}{There exist situations where it is possible to perform an index only scan, but it is not possible to perform an index scan.}
\question\TFQuestion{T}{There exist situations where it is possible to perform an index scan, but it is not possible to perform an index only scan.}
\question\TFQuestion{F}{When performing an index only scan, postgres only needs to consult the index file, and never needs to consult the table file.}
\question\TFQuestion{T}{Postgres uses a table's VM when performing an index only scan.} 
\question\TFQuestion{F}{Postgres indexes contain enough metainformation for each tuple in order to determine the tuple's visibility.}
\question\TFQuestion{F}{In some situations, an index scan can access fewer table pages than a bitmap scan.}
\question\TFQuestion{T}{For databases stored on HDDs, the query planner will choose to perform sequential scans instead of index scans relatively more often than when the database is stored on SSDs.}
\question\TFQuestion{F}{When inserting large amounts of data into an empty table, it is faster to first create your indexes, then insert the data.}
\question\TFQuestion{T}{When performing an index scan on a BTree Index, the number of comparison operations performed is always less than or equal to the number of pages accessed.}
\question\TFQuestion{T}{Reducing the value of \lstinline{random_page_cost} system parameter will increase the number of situations where the query planner will use an index only/index/bitmap scan over a seq scan.}

%%%%%%%%%%%%%%%%%%%%%%%%%%%%%%%%%%%%%%%%%%%%%%%%%%%%%%%%%%%%%%%%%%%%%%%%%%%%%%%%
\fullwidth{\textbf{Sorting / Grouping Algorithms}}
\question\TFQuestion{T}{A BTree index can be used to speed up SELECT statements with the ORDER BY clause.}
\question\TFQuestion{T}{The query planner will typically prefer an index scan over a bitmap index scan on SELECT queries that use a LIMIT clause with a small value.}
\question\TFQuestion{F}{If a SELECT statement requires an explicit SORT operation in the query plan, then adding a LIMIT clause to the SELECT statement is likely to significantly improve performance.}
\question\TFQuestion{T}{A GROUP BY clause can always be implemented with either a GroupAggregate or a HashAggregate.}
\question\TFQuestion{T}{A BTree index can be used to speed up a GroupAggregate operation.}
\question\TFQuestion{F}{A BTree index can be used to speed up a HashAggregate operation.}
\question\TFQuestion{F}{The HashAggregate requires less memory than the GroupAggregate.}
\question\TFQuestion{T}{Increasing the \lstinline{work_mem} parameter will cause the query planner to more likely prefer a HashAggregate operation.}
\question\TFQuestion{T}{Increasing the \lstinline{work_mem} parameter too high can cause the operating system to unexpectedly kill worker processes.}
\question\TFQuestion{F}{If the \lstinline{work_mem} parameter is lower than the amount of memory needed to complete a computation, the process will be killed by the OS.}
\question\TFQuestion{F}{The HashAggregate algorithm can be used if one of the SELECT columns contains COUNT(DISTINCT *).}

%%%%%%%%%%%%%%%%%%%%%%%%%%%%%%%%%%%%%%%%%%%%%%%%%%%%%%%%%%%%%%%%%%%%%%%%%%%%%%%%
\fullwidth{\textbf{Join Strategies}}
\question\TFQuestion{F}{An index scan can be used to compute the join between two tables.}
\question\TFQuestion{T}{It's always possible to use a nested loop join.}
\question\TFQuestion{T}{It's always possible to use a hash join.}
\question\TFQuestion{F}{It's always possible to use a merge join.}
\question\TFQuestion{T}{The nested loop join can implement joins using a $<$ condition.}
\question\TFQuestion{F}{The hash join can implement joins using a $<$ condition.}
\question\TFQuestion{F}{The merge join can implement joins using a $<$ condition.}
\question\TFQuestion{T}{The nested loop join can implement self joins.}
\question\TFQuestion{T}{The hash join can implement self joins.}
\question\TFQuestion{T}{The merge join can implement self joins.}
\question\TFQuestion{T}{All three join algorithms (nested loop, hash, and merge) can implement joins using on an equality condition.}
\question\TFQuestion{F}{When it's posible to do both a hash join and a merge join, the hash join will always be faster.}
\question\TFQuestion{F}{When it's posible to do both a hash join and a merge join, the merge join will always be faster.}
\question\TFQuestion{T}{A SQL full outer join can be computed using either the nested loop join, hash join, or merge join algorithms.}
\question\TFQuestion{T}{The order that the query planner chooses to join tables together can have a significant impact on the runtime of the join.}
\question\TFQuestion{T}{Increasing the size of \lstinline{work_mem} can improve the performance of a hash join.}
\question\TFQuestion{F}{Appropriately created indexes can speed up hash joins.}

%%%%%%%%%%%%%%%%%%%%%%%%%%%%%%%%%%%%%%%%%%%%%%%%%%%%%%%%%%%%%%%%%%%%%%%%%%%%%%%%

\fullwidth{\textbf{The \lstinline{CLUSTER} command}}

\question\TFQuestion{T}{The CLUSTER command can greatly speed up bitmap scans by reducing the number of table pages accessed.}
\question\TFQuestion{F}{When a table has been CLUSTERed on an index, inserting new tuples causes them to be inserted in the order specified by the index.}
\question\TFQuestion{F}{You can insert into a table while the CLUSTER command is being run.}
\question\TFQuestion{T}{You can insert into a table while the CREATE INDEX CONCURRENTLY command is being run.}
\question\TFQuestion{F}{You can insert into a table while the CREATE INDEX command (without the CONCURRENTLY option) is being run.}
\question\TFQuestion{F}{The \lstinline{maintenance_work_mem} system parameter should be set to a low value in order to make CLUSTER run faster.}
\question\TFQuestion{T}{It is always recommended to run the ANALYZE command after running the CLUSTER command.}

%%%%%%%%%%%%%%%%%%%%%%%%%%%%%%%%%%%%%%%%%%%%%%%%%%%%%%%%%%%%%%%%%%%%%%%%%%%%%%%%

\fullwidth{\textbf{The \lstinline{ANALYZE} command}}

\question\TFQuestion{T}{Running the ANALYZE command on a table helps the query planner choose which scan algorithm to implement.}
\question\TFQuestion{F}{The ANALYZE command should be run after every INSERT command for optimal performance.}
\question\TFQuestion{T}{The ANALYZE command should be run after large bulk inserts for optimal performance.}
\question\TFQuestion{F}{The ANALYZE command is never run automatically.}
\question\TFQuestion{F}{If a database table hasn't changed, but we've created several new indexes on the table, we should run the ANALYZE command so that the query planner knows how to best use those indexes.}
\question\TFQuestion{F}{If a database table has changed because a significant fraction of rows have been inserted, but we have not created any new indexes, then the ANALYZE command will not do anything.}
\question\TFQuestion{T}{Running the ANALYZE command on the tables that are used in a sequence of JOINs can help the query planner choose which order to perform the joins in.}

%%%%%%%%%%%%%%%%%%%%%%%%%%%%%%%%%%%%%%%%%%%%%%%%%%%%%%%%%%%%%%%%%%%%%%%%%%%%%%%%

\fullwidth{\textbf{Constraints}}
\question\TFQuestion{F}{It is possible to have a UNIQUE constraint without an index.}
\question\TFQuestion{T}{It is possible to have a CHECK constraint without an index.}
\question\TFQuestion{T}{It is possible to have a NOT NULL constraint without an index.}
\question\TFQuestion{F}{It is possible to have a FOREIGN KEY constraint without an index on the target table/column(s).}
\question\TFQuestion{F}{It is possible to have a FOREIGN KEY constraint without an index on the source table/column(s).}

%%%%%%%%%%%%%%%%%%%%%%%%%%%%%%%%%%%%%%%%%%%%%%%%%%%%%%%%%%%%%%%%%%%%%%%%%%%%%%%%
\fullwidth{
\textbf{Parallelism}

The questions below all refer specifically to Postgres version 13.
%See \url{https://www.postgresql.org/docs/13/parallel-plans.html} and \url{https://www.postgresql.org/docs/13/how-parallel-query-works.html} extensive details at \url{https://wiki.postgresql.org/wiki/Parallel_Internal_Sort}
}

\question\TFQuestion{F}{It is always more efficient to run a parallelized operation than an unparallelized one, when both methods are available.}
\question\TFQuestion{T}{Sequential scans can be parallelized.}
\question\TFQuestion{T}{Index only scans can be parallelized.}
\question\TFQuestion{T}{Index scans can be parallelized.}
\question\TFQuestion{T}{Bitmap scans can be parallelized.}
\question\TFQuestion{T}{Both the inner and outer sides of a nested loop join can be parallelized.}
\question\TFQuestion{F}{Both the inner and outer sides of a merge join can be parallelized.}
\question\TFQuestion{F}{Both the inner and outer sides of a hash join can be parallelized.}
\question\TFQuestion{F}{For all join strategies, the inner side can be parallelized.}
\question\TFQuestion{T}{For all join strategies, the outer side can be parallelized.}
\end{questions}

%%%%%%%%%%%%%%%%%%%%%%%%%%%%%%%%%%%%%%%%%%%%%%%%%%%%%%%%%%%%%%%%%%%%%%%%%%%%%%%%

%\fullwidth{\textbf{Questions you'll have to read the docs for}}
%
%\question\TFQuestion{T}{Assuming you have sufficient physical RAM, you can increase the size of the \lstinline{maintenance_work_mem} system parameter in order to make creating indexes from scratch faster.}
%\question\TFQuestion{T}{Assuming you have sufficient physical RAM, you can increase the size of the \lstinline{work_mem} system parameter in order to make creating indexes from scratch faster.}
%1. `shared_buffers` is the total number of pages that will be cached in memory
    %1. general caching detail: https://severalnines.com/database-blog/overview-caching-postgresql
    %1. redfin engineering: https://redfin.engineering/how-to-boost-postgresql-cache-performance-8db383dc2d8f
    %1. most details: https://madusudanan.com/blog/understanding-postgres-caching-in-depth/
%1. postgres docs for all parameters: https://www.postgresql.org/docs/current/runtime-config-resource.html#GUC-SHARED-BUFFERS
%\question\TFQuestion{}{}

%%%%%%%%%%%%%%%%%%%%%%%%%%%%%%%%%%%%%%%%%%%%%%%%%%%%%%%%%%%%%%%%%%%%%%%%%%%%%%%%
\newpage
\section{Integrated Questions}

Consider the following simplified normalized twitter schema.
\begin{lstlisting}
CREATE TABLE users (
    id_users BIGINT PRIMARY KEY,
    created_at TIMESTAMPTZ,
    username TEXT
);

CREATE TABLE tweets (
	id_tweets BIGINT PRIMARY KEY,
	id_users BIGINT REFERENCES users(id_users),
	in_reply_to_user_id BIGINT REFERENCES users(id_users),
	created_at TIMESTAMPTZ,
	text TEXT
);

CREATE TABLE tweets_mentions (
	id_tweets BIGINT REFERENCES tweets(id_tweets),
	id_users BIGINT REFERENCES users(id_users),
	PRIMARY KEY (id_tweets, id_users)
);
\end{lstlisting}

\begin{questions}
\question{
List all the tables/columns that have indexes created on them.
}

\newpage
\question{
List the scan methods applicable for the following SQL query.

\lstinline{SELECT count(*) FROM tweets WHERE id_user=:id_user;}

}
\vspace{1.5in}

\question{
List the scan methods applicable for the following SQL query.

\lstinline{SELECT id_users,username FROM users WHERE id_user=:id_user;}

}
\vspace{1.5in}

\question{
Explain why the following SQL query is likely to be inefficient,
and create an index that will speed up the query.

\begin{lstlisting}
SELECT id_tweets
FROM tweets_mentions
WHERE id_users=:id_users;
\end{lstlisting}
}
\vspace{1.5in}

\question{
Create index(es) so that the following query could use an index only scan.

Do not create any unneeded indexes; if no new indexes are needed, say so.

\begin{lstlisting}
SELECT count(*)
FROM tweets
WHERE id_user=:id_user 
  AND created_at < :hi 
  AND created_at >= :lo;
\end{lstlisting}

}
\vspace{1.5in}

\question{
Create index(es) so that the following query can use an index only scan, avoid an explicit sort, and take advantage of the LIMIT clause for faster processing.

Do not create any unneeded indexes; if no new indexes are needed, say so.

\begin{lstlisting}
SELECT id_tweets, created_at
FROM tweets
WHERE id_users=:is_users
ORDER BY created_at DESC
LIMIT 10;
\end{lstlisting}
}
\vspace{1.5in}

\question{
Construct index(es) so that the following query can use an index only scan, and the \lstinline{users(username)} column will have a UNIQUE constraint.

Do not create any unneeded indexes; if no new indexes are needed, say so.

\begin{lstlisting}
SELECT created_at FROM users WHERE username=:username;
\end{lstlisting}
}
\vspace{1.5in}

\question{
Construct a single index so that the following query can be answered as quickly as possible.
\begin{lstlisting}
SELECT id_tweets
FROM tweets
WHERE id_user=:id_user
  AND created_at >= '2020-01-01 00:00:00'
  AND created_at <  '2021-01-01 00:00:00'
ORDER BY 
  created_at ASC,
  id_reply_to_user_id DESC
\end{lstlisting}
}
\vspace{1.5in}

\question{
Construct index(es) to speed up the following JOIN,
assuming a merge join is used.

Do not create any unneeded indexes; if no new indexes are needed, say so.

\begin{lstlisting}
SELECT *
FROM tweets_mentions
JOIN tweets USING (id_tweets);
\end{lstlisting}
}
\vspace{2in}


\question{
Construct index(es) to speed up the following JOIN,
assuming a merge join is used.
Your index(es) should take advantage of the WHERE clause.

Do not create any unneeded indexes; if no new indexes are needed, say so.

\begin{lstlisting}
SELECT id_tweets
FROM tweets_mentions
JOIN users USING (id_users)
WHERE username=:username;
\end{lstlisting}
}
\vspace{1.5in}

\newpage
\question{
Your goal is to answer the following query quickly.
\begin{lstlisting}
SELECT count(*)
FROM tweets
WHERE id_user=:id_user
  AND created_at >= '2020-01-01 00:00:00';
\end{lstlisting}
}
You have the option of creating either of the following two indexes:
\begin{lstlisting}
CREATE INDEX tweets_idx1 ON tweets(id_user)
	WHERE created_at >= '2020-01-01 00:00:00';
CREATE INDEX tweets_idx2 ON tweets(id_user,created_at);
\end{lstlisting}
If the tweets table is large (several TBs), which index will result in the fewest page reads when answering the SELECT query?  Why?
\vspace{1.5in}

Which index will use the least amount of disk space?
\vspace{1.5in}

\newpage
\question{
You are considering adding more information to the tweets table by redefining the schema as:
\begin{lstlisting}
CREATE TABLE tweets (
    id_tweets BIGINT PRIMARY KEY,
    id_users BIGINT REFERENCES users(id_users),
    created_at TIMESTAMPTZ,
    in_reply_to_status_id BIGINT,
    in_reply_to_user_id BIGINT REFERENCES users(id_users),
    quoted_status_id BIGINT,
    retweet_count SMALLINT,
    favorite_count SMALLINT,
    quote_count SMALLINT,
    withheld_copyright BOOLEAN,
    withheld_in_countries VARCHAR(2)[],
    source TEXT,
    text TEXT,
    country_code VARCHAR(2),
    state_code VARCHAR(2),
    lang TEXT,
    place_name TEXT,
    geo geometry
);
\end{lstlisting}
How will this affect the number of pages accessed during (and therefore the runtime of) a
\begin{enumerate}
\item sequential scan?
\vspace{0.5in}
\item index only scan?
\vspace{0.5in}
\item index scan?
\vspace{0.5in}
\item bitmap scan?
\vspace{0.5in}
\end{enumerate}

You should also understand (but do not need to explain) why the number of pages accessed during a GroupAggregate/HashAggregate or Nested Loop/Hash/Merge join are not affected.
}


\end{questions}
\end{document}



