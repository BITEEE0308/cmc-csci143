\documentclass[10pt]{article}

\usepackage[margin=1in]{geometry}
\usepackage{amsmath}
\usepackage{amssymb}
\usepackage{amsthm}
\usepackage{mathtools}
\usepackage[shortlabels]{enumitem}
\usepackage[normalem]{ulem}

\usepackage{hyperref}
\hypersetup{
  colorlinks   = true, %Colours links instead of ugly boxes
  urlcolor     = black, %Colour for external hyperlinks
  linkcolor    = blue, %Colour of internal links
  citecolor    = blue  %Colour of citations
}

\usepackage{courier}
\usepackage{listings}
\lstset{numbers=left}
\lstset{basicstyle=\ttfamily}

%%%%%%%%%%%%%%%%%%%%%%%%%%%%%%%%%%%%%%%%%%%%%%%%%%%%%%%%%%%%%%%%%%%%%%%%%%%%%%%%

\theoremstyle{definition}
\newtheorem{problem}{Problem}
\newcommand{\E}{\mathbb E}
\newcommand{\R}{\mathbb R}
\DeclareMathOperator{\Var}{Var}
\DeclareMathOperator*{\argmin}{arg\,min}
\DeclareMathOperator*{\argmax}{arg\,max}

\newcommand{\trans}[1]{{#1}^{T}}
\newcommand{\loss}{\ell}
\newcommand{\w}{\mathbf w}
\newcommand{\mle}[1]{\hat{#1}_{\textit{mle}}}
\newcommand{\map}[1]{\hat{#1}_{\textit{map}}}
\newcommand{\normal}{\mathcal{N}}
\newcommand{\x}{\mathbf x}
\newcommand{\y}{\mathbf y}
\newcommand{\ltwo}[1]{\lVert {#1} \rVert}

%%%%%%%%%%%%%%%%%%%%%%%%%%%%%%%%%%%%%%%%%%%%%%%%%%%%%%%%%%%%%%%%%%%%%%%%%%%%%%%%

\begin{document}

\begin{center}
{
\Large
    Quiz: Shell
}

    \vspace{0.1in}
\end{center}


\vspace{0.15in}
\noindent
\textbf{Total Score:} ~~~~~~~~~~~~~~~/12

\vspace{0.5in}
\noindent
\textbf{Printed Name:}

\noindent
\rule{\textwidth}{0.1pt}
\vspace{0.25in}

\noindent
\textbf{Rules:}
\begin{enumerate}
    \item You MAY use any printed or handwritten notes.
    \item You MAY NOT use a computer or any other electronic device.
    \item You MAY NOT discuss this quiz with any other human.
\end{enumerate}

\vspace{0.25in}
\noindent
\textbf{Instructions:}
\begin{enumerate}
    \item Attached is a POSIX-compliant shell script.
        Some of these commands have output, and some of them do not.
        There are six lines that end in the command \lstinline{| wc -l},
        and your job is to write the output from those lines.
    \item There are 6 problems.
        Each correct problem is worth 2 points,
        each blank problem is worth 0 points,
        \textbf{each incorrectly answered problem is worth -1 point}.
\end{enumerate}

\vspace{0.25in}
\noindent
\textbf{Problem formation rules:}
There are several practice shell scripts in the files \lstinline{example?.sh}.
They all follow the same format that your quiz will follow.
I will use the following transformations to generate new shell scripts for the quiz based on these practice scripts.
\begin{enumerate}
    \item Add/remove \lstinline{git add}, \lstinline{git commit} and \lstinline{git checkout} commands
    \item Change file/branch names
    \item Change redirection and pipe symbols
    \item Add/remove single/double quotes/backticks/dollar signs
    \item Add/remove the -a flag to ls
    \item Add/remove wildcard characters
    \item Add/remove variable references
\end{enumerate}

\newpage

\begin{lstlisting}
#!/bin/sh
set -e

# create a temporary directory for all work to happen in
temp_dir=$(mktemp -d)
cd "$temp_dir"
pwd


echo 'problem 1'
echo big data > .README.md
echo "big data" >> .README.md
echo 'big data' > .README.md
cat .README.md | wc -l



echo 'problem 2'
git init > /dev/null 2> /dev/null
git add .README.md
git commit -m 'first commit' > /dev/null 2> /dev/null
ls -a | wc -l



echo 'problem 3'
git checkout -b new_branch > /dev/null 2> /dev/null
echo test >> .new_file
touch example
git add *
git commit -m 'new_branch' > /dev/null 2> /dev/null
git checkout master > /dev/null 2> /dev/null
ls -a | wc -l



echo 'problem 4'
mkdir dir
for file in a b c d; do echo 'hello world' > dir/$file; done
ls -a dir | wc -l



echo 'problem 5'
var='this is an example'
for file in $var; do echo $file; done | wc -l



echo 'problem 6'
cd dir
touch "*"
rm *
ls -a | wc -l
\end{lstlisting}
\end{document}
