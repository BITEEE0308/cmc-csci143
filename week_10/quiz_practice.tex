\documentclass[10pt]{article}

\usepackage[margin=1in]{geometry}
\usepackage{amsmath}
\usepackage{amssymb}
\usepackage{amsthm}
\usepackage{mathtools}
\usepackage[shortlabels]{enumitem}
\usepackage[normalem]{ulem}
\usepackage{booktabs}
\usepackage{tcolorbox}

\usepackage{hyperref}
\hypersetup{
  colorlinks   = true, %Colours links instead of ugly boxes
  urlcolor     = black, %Colour for external hyperlinks
  linkcolor    = blue, %Colour of internal links
  citecolor    = blue  %Colour of citations
}

\usepackage{courier}
\usepackage{listings}
\lstset{numbers=left}
\lstset{basicstyle=\ttfamily}

%%%%%%%%%%%%%%%%%%%%%%%%%%%%%%%%%%%%%%%%%%%%%%%%%%%%%%%%%%%%%%%%%%%%%%%%%%%%%%%%

\theoremstyle{definition}
\newtheorem{problem}{Problem}
\newcommand{\E}{\mathbb E}
\newcommand{\R}{\mathbb R}
\DeclareMathOperator{\Var}{Var}
\DeclareMathOperator*{\argmin}{arg\,min}
\DeclareMathOperator*{\argmax}{arg\,max}

\newcommand{\trans}[1]{{#1}^{T}}
\newcommand{\loss}{\ell}
\newcommand{\w}{\mathbf w}
\newcommand{\mle}[1]{\hat{#1}_{\textit{mle}}}
\newcommand{\map}[1]{\hat{#1}_{\textit{map}}}
\newcommand{\normal}{\mathcal{N}}
\newcommand{\x}{\mathbf x}
\newcommand{\y}{\mathbf y}
\newcommand{\ltwo}[1]{\lVert {#1} \rVert}

\newcounter{ProblemCounter}
\newcommand{\nextproblem}{\stepcounter{ProblemCounter} \noindent\textbf{Problem \theProblemCounter}:}

%%%%%%%%%%%%%%%%%%%%%%%%%%%%%%%%%%%%%%%%%%%%%%%%%%%%%%%%%%%%%%%%%%%%%%%%%%%%%%%%

\begin{document}

\begin{center}
{
\Large
    (Practice) Quiz: Transactions and Locks
}

    \vspace{0.1in}
\end{center}

\vspace{0.15in}
\noindent
\textbf{Total Score:} ~~~~~~~~~~~~~~~/15

\vspace{0.5in}
\noindent
\textbf{Printed Name:}

\noindent
\rule{\textwidth}{0.1pt}
\vspace{0.25in}

\noindent
\textbf{Instructions:}
\begin{enumerate}
    \item Each problem is worth 3 points.
    \item For each problem:
        \begin{enumerate}
        \item Circle all SQL commands that block.
        \item Write an X through all commands that error, even if the command previously blocked.
        \item Write the output of each \lstinline{SELECT} statement.
            (You do not need to write column names, and the order of rows does not matter.)
        \item If the sequence of commands results in a deadlock, say so.
        \end{enumerate}
\end{enumerate}

\vspace{0.2in}
\noindent
\textbf{Rules:}
\begin{enumerate}
    \item You MAY:
        \begin{enumerate}
            \item Use any handwritten/printed notes
            \item Use your computer
            \item Reference arbitrary written material on the internet, including the Postgres documentation
        \end{enumerate}
    \item You MAY NOT:
        \begin{enumerate}
            \item Communicate with another human being
            \item Connect to postgres and run commands
        \end{enumerate}
\end{enumerate}

\newpage
%%%%%%%%%%%%%%%%%%%%%%%%%%%%%%%%%%%%%%%%
\nextproblem
\begin{tcolorbox}
\begin{tabular}{p{3.25in}p{3in}}
\hspace{-0.2in}Session 1
\begin{lstlisting}
CREATE TABLE t (a INT UNIQUE);
CREATE TABLE u (
    b INT REFERENCES t(a)
    DEFERRABLE INITIALLY DEFERRED
    );
INSERT INTO t VALUES (1);
INSERT INTO t VALUES (2);
INSERT INTO t VALUES (3);
INSERT INTO u VALUES (1);
INSERT INTO u VALUES (2);
BEGIN;

DELETE FROM u WHERE b=1;
DELETE FROM u WHERE b=3;

INSERT INTO t VALUES (4);

COMMIT;
SELECT count(*) FROM u;
\end{lstlisting}
    &
\hspace{-0.2in}Session 2
\begin{lstlisting}











BEGIN;


INSERT INTO u VALUES (4);

INSERT INTO u VALUES (1);


COMMIT;
\end{lstlisting}
\end{tabular}
\vspace{-0.2in}
\end{tcolorbox}

%%%%%%%%%%%%%%%%%%%%%%%%%%%%%%%%%%%%%%%%
\newpage
\nextproblem
\begin{tcolorbox}
\begin{tabular}{p{3.25in}p{3in}}
\hspace{-0.2in}Session 1
\begin{lstlisting}


BEGIN;
LOCK TABLE t IN EXCLUSIVE MODE;
  


CREATE INDEX ON u(a);

COMMIT;
\end{lstlisting}
    &
\hspace{-0.2in}Session 2
\begin{lstlisting}
CREATE TABLE t ( a INT );
CREATE TABLE u ( a INT UNIQUE );


BEGIN;
LOCK TABLE u IN EXCLUSIVE MODE;
INSERT INTO t VALUES (5);

COMMIT;
\end{lstlisting}
\end{tabular}
\end{tcolorbox}
\vspace{2.5in}

%%%%%%%%%%%%%%%%%%%%%%%%%%%%%%%%%%%%%%%%
\newpage
\nextproblem
\begin{tcolorbox}
\begin{tabular}{p{3.25in}p{3in}}
\hspace{-0.2in}Session 1
\begin{lstlisting}
CREATE TABLE t ( a INT );
INSERT INTO t VALUES (1);
INSERT INTO t VALUES (2);
INSERT INTO t VALUES (3);

BEGIN;
UPDATE t SET a=6 WHERE a=1;

INSERT INTO t VALUES (0);
COMMIT;


SELECT count(*) FROM t;
\end{lstlisting}
    &
\hspace{-0.2in}Session 2
\begin{lstlisting}




BEGIN;


UPDATE t SET a=7 WHERE a=1;


DELETE FROM t WHERE a=6;
COMMIT;

\end{lstlisting}
\end{tabular}
\vspace{-0.2in}
\end{tcolorbox}
\vspace{2.5in}

%%%%%%%%%%%%%%%%%%%%%%%%%%%%%%%%%%%%%%%%
\newpage
\nextproblem
\begin{tcolorbox}
\begin{tabular}{p{3.25in}p{3in}}
\hspace{-0.2in}Session 1
\begin{lstlisting}
CREATE TABLE t (
    a INT UNIQUE 
    DEFERRABLE INITIALLY DEFERRED
    );
INSERT INTO t VALUES (1);
INSERT INTO t VALUES (2);
INSERT INTO t VALUES (3);

BEGIN;
LOCK TABLE t IN ROW SHARE MODE;
UPDATE t SET a=6 WHERE a=1;
UPDATE t SET a=6 WHERE a=2;


COMMIT;
SELECT * FROM t;
\end{lstlisting}
    &
\hspace{-0.2in}Session 2
\begin{lstlisting}







BEGIN;




UPDATE t SET a=7 WHERE a=2;
LOCK TABLE t IN SHARE MODE;


COMMIT;

\end{lstlisting}
\end{tabular}
\vspace{-0.2in}
\end{tcolorbox}
\vspace{2.5in}

%%%%%%%%%%%%%%%%%%%%%%%%%%%%%%%%%%%%%%%%
\newpage
\nextproblem
\begin{tcolorbox}
\begin{tabular}{p{3.25in}p{3in}}
\hspace{-0.2in}Session 1
\begin{lstlisting}
CREATE TABLE t ( a INT UNIQUE );
INSERT INTO t VALUES (1);
INSERT INTO t VALUES (2);
INSERT INTO t VALUES (3);



BEGIN;
INSERT INTO t VALUES (5);
INSERT INTO t VALUES (6);

INSERT INTO t VALUES (2);
COMMIT;
  
\end{lstlisting}
    &
\hspace{-0.2in}Session 2
\begin{lstlisting}




BEGIN ISOLATION LEVEL
REPEATABLE READ;
SELECT count(*) FROM t;



SELECT count(*) FROM t;


SELECT count(*) FROM t;
COMMIT;

\end{lstlisting}
\end{tabular}
\vspace{-0.2in}
\end{tcolorbox}
\vspace{2.5in}


\end{document}

