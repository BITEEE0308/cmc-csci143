\documentclass[10pt]{article}

\usepackage[margin=1in]{geometry}
\usepackage{amsmath}
\usepackage{amssymb}
\usepackage{amsthm}
\usepackage{mathtools}
\usepackage[shortlabels]{enumitem}
\usepackage[normalem]{ulem}
\usepackage{courier}

\usepackage{hyperref}
\hypersetup{
colorlinks   = true, %Colours links instead of ugly boxes
urlcolor     = black, %Colour for external hyperlinks
linkcolor    = blue, %Colour of internal links
citecolor    = blue  %Colour of citations
}

\usepackage[T1]{fontenc}
\usepackage{upquote}
\usepackage{listings}
\lstset{
language=HTML
,basicstyle=\linespread{1}\ttfamily
,keywordstyle=
,language=sh
,showstringspaces=false
,numbers=left
,breaklines=true
}


%%%%%%%%%%%%%%%%%%%%%%%%%%%%%%%%%%%%%%%%%%%%%%%%%%%%%%%%%%%%%%%%%%%%%%%%%%%%%%%%

\theoremstyle{definition}
\newtheorem{problem}{Problem}
\newtheorem{note}{Note}
\newcommand{\E}{\mathbb E}
\newcommand{\R}{\mathbb R}
\DeclareMathOperator{\Var}{Var}
\DeclareMathOperator*{\argmin}{arg\,min}
\DeclareMathOperator*{\argmax}{arg\,max}

\newcommand{\trans}[1]{{#1}^{T}}
\newcommand{\loss}{\ell}
\newcommand{\w}{\mathbf w}
\newcommand{\mle}[1]{\hat{#1}_{\textit{mle}}}
\newcommand{\map}[1]{\hat{#1}_{\textit{map}}}
\newcommand{\normal}{\mathcal{N}}
\newcommand{\x}{\mathbf x}
\newcommand{\y}{\mathbf y}
\newcommand{\ltwo}[1]{\lVert {#1} \rVert}

%%%%%%%%%%%%%%%%%%%%%%%%%%%%%%%%%%%%%%%%%%%%%%%%%%%%%%%%%%%%%%%%%%%%%%%%%%%%%%%%

\begin{document}
\begin{center}
{
\Large
Quiz: POSIX Shell II
}

\vspace{0.1in}
\end{center}

\vspace{0.15in}
\noindent
\textbf{Total Score:} ~~~~~~~~~~~~~~~/$2^2$

\vspace{0.2in}
\noindent
\textbf{Printed Name:}

\noindent
\rule{\textwidth}{0.1pt}
\vspace{0.15in}

\noindent
\textbf{Quiz rules:}
\begin{enumerate}
\item You MAY use any printed or handwritten notes.
\item You MAY NOT use a computer or any other electronic device.
\end{enumerate}

\noindent

\vspace{0.15in}

\filbreak
\begin{problem}
Write the output of the final command in the following terminal session.
If the command has no output, then leave the problem blank.
\end{problem}
\begin{lstlisting}
$ cd; rm -rf quiz; mkdir quiz; cd quiz
$ foo='$(echo hello)'
$ cat > quiz.sh <<EOF
foo='hola'
if [ '$foo' = 'hello' ]; then
    touch if
elif [ '$foo' = 'hola' ]; then
    touch elif
else
    touch else
fi
EOF
$ sh quiz.sh
$ ls
\end{lstlisting}

\vspace{1in}

\filbreak
\begin{problem}
    Write the output of the final command in the following terminal session.
    If the command has no output, then leave the problem blank.
\end{problem}
\begin{lstlisting}
$ cd; rm -rf quiz; mkdir quiz; cd quiz
$ foo='hola'
$ echo 'hello' > foo
$ ( [ "$foo" = 'hello' ] && echo $foo ) >> foo
$ cat foo
\end{lstlisting}

\filbreak
\begin{problem}
    Write the output of the final command in the following terminal session.
    If the command has no output, then leave the problem blank.
\end{problem}
\begin{lstlisting}
$ cd; rm -rf quiz; mkdir quiz; cd quiz
$ cat > logs <<EOF
INFO: blah
INFO: blah
WARNING: blah blah blah
INFO: blah
EOF
$ cat > quiz.sh <<'EOF'
if cat logs | grep ERROR > /dev/null; then
    touch error
elif cat logs | grep WARNING > /dev/null; then
    touch warning
elif cat logs | grep INFO > /dev/null; then
    touch info
fi
EOF
$ sh quiz.sh
$ ls
\end{lstlisting}
\vspace{1in}

\filbreak
\begin{problem}
    Write the output of the final command in the following terminal session.
    If the command has no output, then leave the problem blank.
\end{problem}
\begin{lstlisting}
$ cd; rm -rf quiz; mkdir quiz; cd quiz
$ foo='hola'
$ cat > quiz.sh <<'EOF'
foo='salve'
bar='hello'
if ! false && [ $foo = hola ] || ! [ $bar = salve ]; then
    touch if
else
    touch else
fi
EOF
$ sh quiz.sh
$ ls
\end{lstlisting}

\end{document}
